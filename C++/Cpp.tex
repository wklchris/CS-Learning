\documentclass[a4paper, zihao=-4, linespread=1]{ctexrep}
% Typesetting Formats
\setlength{\lineskiplimit}{3pt}
\setlength{\lineskip}{3pt}

\usepackage{geometry}
  \geometry{vmargin=1.1in, hmargin=1.2in}

%\usepackage{titlesec}
%\titleclass{\part}{top}
%\titleformat{\part}[display]{\huge\bfseries\centering}
%    {\partname~\thepart}{0pt}{}
%\titlespacing*{\part}{0pt}{40pt}{40pt}


% Lists
\usepackage[normalem]{ulem}
\usepackage[inline]{enumitem}
	\setlist[enumerate]{font=\bfseries,labelindent=0pt,itemsep=0pt,parsep=0pt,topsep=.5ex,partopsep=0pt}
	\setlist[itemize]{font=\bfseries,itemsep=0pt,parsep=0pt,topsep=.5ex,partopsep=0pt}
	\setlist[description]{font=\bfseries\ttfamily,itemsep=0pt,parsep=.5ex,topsep=0pt,partopsep=0pt}


% Maths & Plotting
\usepackage{amsmath}

\usepackage{graphicx}
  \graphicspath{{./pics/}}
\usepackage{standalone}
\usepackage{tikz}
  \usetikzlibrary{math}

% Links
\usepackage{hyperref}
\hypersetup{colorlinks, bookmarksopen = true, bookmarksnumbered = true, pdftitle=CS-Learning, pdfauthor=K.L Wu, pdfstartview=FitH}


\endinput


\newcommand\cpp{\texttt{C++}}
\NewDocumentCommand{\cppline}{v}{{\lstinline[language=C++]{#1}}}
\newcommand\cpplib[1]{\textcolor{red!80!black}{\bfseries #1}}

\newcommand\hlt[1]{\textcolor{red}{\bfseries #1}}
\newcommand\cppkw[1]{\textcolor{black}{\bfseries #1}}

% Document Info
\title{\cpp\ 入门}
\author{
  本系列笔记托管于 \href{https://github.com/wklchris/CS-Learning}{CS-Learning} Github 仓库\\
  by \href{email:wklchris@hotmail.com}{wklchris}
}
\date{更新于:~\today}

\begin{document}
\maketitle

\chapter{入门}
\cpp\ 是一种面向对象的高级语言,大部分继承自 \texttt{C} 语言.本手册希望读者有编程基础.

\section{数据类型}
\begin{description}
\item[int] 整型.
\end{description}

\section{基础操作符}
\cpp\ 中的基础操作符有如下:
\begin{description}
\item[赋值符] 等号 \cppline{=} 用于赋值,即将右操作数的值赋给左操作数;它也是大部分主流编程语言的赋值符.
\item[数学运算符] 四则运算符 \cppline{+ - * /}.
\item[复合赋值符] 比如 \cppline{a += b},将右操作数 \cppline{b} 与左操作数 \cppline{a} 相加,再赋值给左操作数;相当于 \cppline{a = a+b}.类似的还有 \cppline{-= *= /=}.
\item[自增符] 比如 \cppline{++i},结果上与 \cppline{i = i + 1} 相同.另一种写法是 \cppline{i++},区别在于 \cppline{a = ++i} 会先自增再赋值,而 \cppline{a = i++} 则相反.虽然两种都适用,但\hlt{通常将自增符写在变量左侧},即 \cppline{++i}.
\item[自减符] 类似,比如 \cppline{--i}.
\end{description}

\section{主函数}
废话不谈,单刀直入.一个最简单的 \cpp 程序:
\begin{cpplang}
int main() {
    return 0;
}
\end{cpplang}
\hlt{\cpp\ 必须有 \texttt{main()} 函数}.

\section{编译}
将上节中的三行代码保存于 \texttt{test.cpp} 文件中,并编译(以 \href{http://www.mingw.org}{\texttt{g++}} 为例):
\begin{lstlisting}
g++ test.cpp -o test
\end{lstlisting}
其中参数 \texttt{-o} 表示编译结果将命名为 \texttt{test.exe}.

\section{标准库:例子}
\cpp\ 本身不提供输入或输出,但其标准库 \cpplib{iostream} 提供了该功能.它支持 4 个输入输出对象,分别是标准输入  \cppline{cin},标准输出 \cppline{cout},标准错误 \cppline{cerr},与日志 \cppline{clog}.

库的加载语句以 \cppline{#include} 开头:
\begin{cpplang}
#include <iostream>
int main() {
    std::cout << "Add two numbers:" << std::endl;
    int a, b;
    std::cin >> a >> b;
    std::cout << a + b << std::endl;
    return 0;
}
\end{cpplang}
其中,尖括号中的 \cppline{iostream} 是\cppkw{头文件}.而 \cppline{<<} 称为\cppkw{输出操作符},会返回左操作数(即 \cppline{std::cout}).类似的,\cppline{>>} 称为\cppkw{输入操作符},也返回左操作数.因为它们都返回左操作数,所以可以连写\footnote{类似于做加法,$(1+2)+4$ 实质是两个加法操作 $1+2=3$ 与 $3+4$ 的连写}.

\cppline{std::endl} 叫 \cppkw{操纵符(manipulator)},在输出流中生成换行,并刷新缓冲区来立刻更新将输出流输出.

上例中的 \cppline{std::cout} 表示命令 \cppline{cout} 来自 \cppline{std} \cppkw{命名空间(namespace)}.如果使用 \cppline{using namespace std;} 命令,就可以不用再指定命名空间 \cppline{std}:“
\begin{cpplang}
#include <iostream>
using namespace std;
int main() {
    cout << "Add two numbers:" << endl;
    int a, b;
    cin >> a >> b;
    cout << a + b << endl;
    return 0;
}
\end{cpplang} 

\section{注释}
\cpp\ 中的注释:
\begin{itemize}
\item 单行注释:以 \cppline{//} 开头,该行位于其后的部分都是注释.
\item 多行注释:以 \cppline{/*} 开头、以 \cppline{*/} 结尾,之间是注释.
\end{itemize}

唯一需要注意的地方是\hlt{多行注释不可嵌套}.因此,在注释多行代码时,编辑器通常的做法是在每一行的行首插入单行注释符号.

\chapter{语法}
\section{控制语句}
\begin{description}
\item[条件语句] \cppline{if} 语句.
\item[循环语句] 包含 \cppline{for} 与 \cppline{while} 两种形式.
\end{description}

\subsection{if 语句}
输出两数中较大数的例子:
\begin{cpplang}
#include <iostream>
int main() {
    int s, a, b;
    std::cout << "Input two numbers:" << std::endl;
    std::cin >> a >> b;
    if (a > b) {
        s = a;
    } else {
        s = b;
    }
    std::cout << s << std::endl;
    return 0;
}
\end{cpplang}

\subsection{for 语句}
输出 $1+2+3$ 计算结果的例子:
\begin{cpplang}
#include <iostream>
int main() {
    int s = 0;
    for (int i = 1; i < 4; ++i) {
        s += i;
    }
    std::cout << s << std::endl;
    return 0;
}
\end{cpplang}
以上循环定义了循环变量 \cppline{i},它将被赋初值 1,然后检测是否满足 $i<4$ 的条件.若满足就执行循环体内容,并将 \cppline{i} 自增 1,再进行检测……直到条件不满足为止.请谨记,\cppline{for} 循环的\hlt{初次循环前就会检查条件是否满足}.

上例中的变量 \cppline{i} 是在循环内声明的,因此在循环外不可再访问.

\subsection{while 语句}
同样是输出 $1+2+3$ 计算结果的例子,不过这次使用 \cppline{while} 循环:
\begin{cpplang}
#include <iostream>
int main() {
    int s = 0, i = 1;
    while (i < 4) {
        s += i;
        ++i;
    }
    std::cout << s << std::endl;
    return 0;
}
\end{cpplang}

\section{类}
\cpp\ 有许多预定义的数据类型,比如整型、布尔型. \cppkw{类(class)}的存在使得我们可以自定义这样的“数据类型”,这也是面向对象编程的精髓.上文介绍的标准库 \cpplib{iostream},其实也是一种类.

本节以扑克牌为例,说明一个类该如何设计、怎样定义,并展示它如何使得代码书写更快速、更易读.

\subsection{类的设计}
对于一副不含大小王的扑克牌,我们很容易发现每张牌的共有特征:每张牌都有一种花色,并且有一个点数值.根据这两个特征,我们就能从一副扑克中精确地找到这张牌.因此,我们的类就需要两个\cppkw{成员}.

如果我们不定义类,而是使用 \cpp\ 预定义的数据类型,怎么表示一副扑克牌?朴素的想法是使用两个长度为 52 的数组,一个是字符串数组 $A$,每个元素记录花色;另一个是整型数组 $B$,每个元素记录点数值.然后,我们将同一个索引值 $i$ 对应的花色与点数结合起来,组成一张花色为 $A[i]$、点数为 $B[i]$ 的扑克.

上述的这种定义方式十分笨拙;而且阅读者很难注意到两个数组间关联的含义,因此代码可读性也会很差.定义一个扑克类会帮助我们改善代码.

\subsection{类的定义}
类的定义方式是在工作目录下建立一个扩展名为 \texttt{.h} 的文件,比如 \texttt{Poker.h}:
\begin{cpplang}
#include <string>
using namespace std;

class Poker {
    int val;
    string s;
public:
    Poker(string suits, int number);  // Contructor
    string stringfy(void) {
        return s + " " + to_string(val);
    }
};

Poker::Poker(string suits, int number) {
    s = suits;
    val = number;
}
\end{cpplang}
注意,类定义语句(\cppline{class} 语句)是\hlt{需要分号结尾}的.

在 \cppline{Poker} 类的定义中,\cppline{public} 以下的行都是公有成员.未指明的(或指明 \cppline{private} 的)则是私有成员,它们无法从类外被访问.

上例中有一个公有函数 \cppline{Poker}, 与该类同名,称为\cppkw{构造函数(constructor)}.该函数在声明一个类的\cppkw{实例(instance)}时会直接被调用,用来完成一些创建实例的操作.

\subsection{类的使用}
在主文件中,使用 \cppline{#include "Poker.h"} 来加载这个类.构造变量 \cppline{a} 时将花色与点数两个参数传入,组成一个“梅花 9”扑克牌.
\begin{cpplang}
#include <iostream>
#include "Poker.h"
using namespace std;

int main() {
    Poker a ("Club", 9);
    cout << a.stringfy() << endl;
    return 0;
}
\end{cpplang}
上例会输出“Club 9”.

\end{document}
